\documentclass{article}
\usepackage{graphicx} % Required for inserting images
\usepackage[utf8]{inputenc}
\usepackage[english,russian]{babel}
\usepackage{indentfirst}
\usepackage{misccorr}
\usepackage{amsmath}
\usepackage{float}
\begin{document}
	\begin{titlepage}
		\begin{center}
			\large
			МИНИСТЕРСТВО НАУКИ И ВЫСШЕГО ОБРАЗОВАНИЯ РОССИЙСКОЙ ФЕДЕРАЦИИ
			
			Федеральное государственное бюджетное образовательное учреждение высшего образования
			
			\textbf{АДЫГЕЙСКИЙ ГОСУДАРСТВЕННЫЙ УНИВЕРСИТЕТ}
			\vspace{0.25cm}
			
			Инженерно-физический факультет
			
			Кафедра автоматизированных систем обработки информации и управления
			\vfill
			
			\vfill
			
			\textsc{Отчет по практике}\\[5mm]
			
			{\LARGE Программаная реализация численного метода \textit{Текст из задания по варианту.}}
			\bigskip
			
			1 курс, группа 1ИВТ АСОИУ
		\end{center}
		\vfill
		
		\newlength{\ML}
		\settowidth{\ML}{«\underline{\hspace{0.7cm}}» \underline{\hspace{2cm}}}
		\hfill\begin{minipage}{0.5\textwidth}
			Выполнил:\\
			\underline{\hspace{\ML}}В.\,А.~Сапунов\\
			«06» \,06. 2024 г.
		\end{minipage}%
		\bigskip
		
		\hfill\begin{minipage}{0.5\textwidth}
			Руководитель:\\
			\underline{\hspace{\ML}} С.\,В.~Теплоухов\\
			«06» \,06. 2024 г.
		\end{minipage}%
		\vfill
		
		\begin{center}
			Майкоп, 2024 г.
		\end{center}
	\end{titlepage}
	\section{Введение}
	\label{sec:intro}
	% Что должно быть во введении
	\begin{enumerate}
		\item Вариант 5 - Вычислить матрицу обратную заданной.
		\item Пример кода, решающего данную задачу
		\item График
		\item Скриншот программы
	\end{enumerate}
	
	
	\section{Ход работы}
	\label{sec:exp}
	
	\subsection{Код приложения}
	\label{sec:exp:code}
	\begin{verbatim}
		#include <iostream>
		#include <vector>
		#include <iomanip>
		#include <limits>
		
		using namespace std;
		
		void printMatrix(const vector<vector<double>>& matrix) {
			int rows = matrix.size();
			int cols = matrix[0].size();
			
			for (int i = 0; i < rows; i++) {
				for (int j = 0; j < cols; j++) {
					cout << setw(8) << matrix[i][j] << " ";
				}
				cout << endl;
			}
		}
		
		vector<vector<double>> inverseMatrix(vector<vector<double>>& matrix) {
			int n = matrix.size();
			
			vector<vector<double>> extendedMatrix(n, vector<double>(2 * n));
			for (int i = 0; i < n; i++) {
				for (int j = 0; j < n; j++) {
					extendedMatrix[i][j] = matrix[i][j];
				}
				extendedMatrix[i][i + n] = 1;
			}
			
			for (int i = 0; i < n; i++) {
				int pivotRow = i;
				while (pivotRow < n && abs(extendedMatrix[pivotRow][i]) < 1e-6) {
					pivotRow++;
				}
				
				if (pivotRow == n) {
					continue;
				}
				
				if (pivotRow != i) {
					swap(extendedMatrix[i], extendedMatrix[pivotRow]);
				}
				
				double pivot = extendedMatrix[i][i];
				for (int j = i; j < 2 * n; j++) {
					extendedMatrix[i][j] /= pivot;
				}
				
				for (int k = i + 1; k < n; k++) {
					double factor = extendedMatrix[k][i];
					for (int j = i; j < 2 * n; j++) {
						extendedMatrix[k][j] -= factor * extendedMatrix[i][j];
					}
				}
			}
			
			for (int i = 0; i < n; i++) {
				bool allZero = true;
				for (int j = 0; j < n; j++) {
					if (abs(extendedMatrix[i][j]) > 1e-6) {
						allZero = false;
						break;
					}
				}
				if (allZero) {
					cout << "Матрица вырождена! Обратная матрица не существует." << endl;
					return {};
				}
			}
			
			for (int i = n - 1; i >= 0; i--) {
				// Обнуление элементов над ведущим элементом
				for (int k = i - 1; k >= 0; k--) {
					double factor = extendedMatrix[k][i];
					for (int j = i; j < 2 * n; j++) {
						extendedMatrix[k][j] -= factor * extendedMatrix[i][j];
					}
				}
			}
			
			vector<vector<double>> inverse(n, vector<double>(n));
			for (int i = 0; i < n; i++) {
				for (int j = n; j < 2 * n; j++) {
					inverse[i][j - n] = extendedMatrix[i][j];
				}
			}
			
			return inverse;
		}
		
		int main() {
			setlocale(0, "ru");
			int choice;
			int n;
			vector<vector<double>> matrix;
			
			do {
				cout << "\nМеню:" << endl;
				cout << "1. Ввести новую матрицу" << endl;
				cout << "2. Найти обратную матрицу" << endl;
				cout << "3. Выход" << endl;
				cout << "Введите ваш выбор: ";
				cin >> choice;
				vector<vector<double>> inverse = inverseMatrix(matrix);
				switch (choice) {
					case 1:
					cout << "Введите размерность матрицы: ";
					cin >> n;
					
					if (n <= 0) {
						cout << "Некорректная размерность матрицы! Введите положительное число." << endl;
						break;
					}
					
					matrix.resize(n, vector<double>(n));
					cout << "Введите элементы матрицы:" << endl;
					for (int i = 0; i < n; i++) {
						for (int j = 0; j < n; j++) {
							if (!(cin >> matrix[i][j])) {
								cout << "Ошибка ввода данных! Введите числовые значения." << endl;
								cin.clear();
								cin.ignore(numeric_limits<streamsize>::max(), '\n');
								break;
							}
						}
					}
					cout << "Исходная матрица:" << endl;
					printMatrix(matrix);
					break;
					
					case 2:
					if (matrix.empty()) {
						cout << "Сначала введите матрицу." << endl;
						break;
					}
					if (inverse.empty()) {
						cout << "Обратная матрица не существует!" << endl;
					}
					else {
						cout << "Обратная матрица:" << endl;
						printMatrix(inverse);
					}
					break;
					
					case 3:
					cout << "Выход из программы." << endl;
					break;
					
					default:
					cout << "Неверный выбор!" << endl;
				}
			} while (choice != 3);
			
			return 0;
		}
		
	\end{verbatim}
	
	\subsection{Метод вычисления обратной матрицы}
	\label{sec:mathexample}
	
	Вычисление обратной матрицы
	([[1. -2.  1.]
	[2.  1. -1.]
	[3.  2. -2.]]):
	\begin{verbatim}
		Алгоритм нахождения обратной матрицы методом исключения неизвестных Гаусса:
		1. К матрице A приписать единичную матрицу того же порядка.
		2. Полученную сдвоенную матрицу преобразовать так, чтобы в левой её части получилась единичная матрица, тогда в правой части на месте единичной матрицы автоматически получится обратная матрица. Матрица A в левой части преобразуется в единичную матрицу путём элементарных преобразований матрицы.
		3. Если в процессе преобразования матрицы A в единичную матрицу в какой-либо строке или в каком-либо столбце окажутся только нули, то определитель матрицы равен нулю, и, следовательно, матрица A будет вырожденной, и она не имеет обратной матрицы. В этом случае дальнейшее нахождение обратной матрицы прекращается.
		
	\end{verbatim}
	
	\section{Изображение с примером нахождения ранга матрицы и результат выполнения программы}
	\label{sec:picexample}
	\begin{figure}[h]
		\centering
		\includegraphics[width=0.7\textwidth]{Primer.jpg}
		\caption{Ранг матрицы}\label{fig:par}
	\end{figure}
	Пример вычисления обратной матрицы представлен на рис.~\ref{fig:par}.
	
	Результат выполнения программы представлен на рис.~\ref{fig:par2}.
	\label{sec:picexample}
	\begin{figure}[H]
		\centering
		\includegraphics[width=0.4\textwidth]{result.jpg}
		\caption{Результат выполнения программы}\label{fig:par2}
	\end{figure}
	
	
	
	\section{Пример библиографических ссылок}
	
	Для изучения «внутренностей» \TeX{} необходимо 
	изучить~\cite{Knuth-2003}, а для использования \LaTeX{} лучше
	почитать~\cite{Lvovsky-2003, Voroncov-2005}.
	
	\begin{thebibliography}{9}
		\bibitem{Knuth-2003}Кнут Д.Э. Всё про \TeX. \newblock --- Москва: Изд. Вильямс, 2003 г. 550~с.
		\bibitem{Lvovsky-2003}Львовский С.М. Набор и верстка в системе \LaTeX{}. \newblock --- 3-е издание, исправленное и дополненное, 2003 г.
		\bibitem{Voroncov-2005}Воронцов К.В. \LaTeX{} в примерах. 2005 г.
	\end{thebibliography}
	
\end{document}